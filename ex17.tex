\chapter{Exercise 17: Heap And Stack Memory Allocation}

In this exercise you're going to make a big leap in difficulty and create
an entire small program to manage a database.  This database isn't very
efficient and doesn't store very much, but it does demonstrate most of
what you've learned so far.  It also introduces memory allocation more
formally and gets you started working with files.  We use some file I/O
functions, but I won't be explaining them too well so you can try to
figure them out first.

As usual, type this whole program in and get it working, then we'll discuss:

\begin{code}{ex17.c}
<< d['code/ex17.c|pyg|l'] >>
\end{code}

TODO: Explain the program....

\section{What You Should See}

You should spend as much time as you can testing that it works, and running
it with \program{Valgrind} to confirm you've got all the memory usage
right.  Here's a session of me testing it normally and then using
\program{Valgrind} to check the operations:

\begin{code}{ex17 output}
\begin{lstlisting}
<< d['code/ex17.out|dexy'] >>
\end{lstlisting}
\end{code}

The actual output of \program{Valgrind} is taken out since you should
be able to detect it.

\begin{aside}{OSX Valgrind "Leaks"}
\program{Valgrind} will report that you're leaking small blocks of memory,
but sometimes it's just over-reporting from OSX's internal APIs.  If you see it
showing leaks that aren't inside your code then just ignore them.
\end{aside}

\section{How To Break It}

This program has a lot of places you can break it, so try some of these
but also come up with your own:

\begin{enumerate}
\item The classic way is to remove some of the safety checks such that you can
    pass in arbitrary data. For example, if you remove the check on line 157
    that prevents you from passing in any record number.
\item You can also try corrupting the data file.  Open it in any editor and
    change random bytes then close it.
\end{enumerate}

\section{Extra Credit}

\begin{enumerate}
\item The \func{die} function needs to be augmented to let you pass the \ident{conn}
    variable so it can close it and clean up.
\item Add more operations you can do on the database, like \ident{find}.
\item Try reworking the program to use a single global for the database connection.
    How does this new version of the program compare to the other one?
\end{enumerate}
