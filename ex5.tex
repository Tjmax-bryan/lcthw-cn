\chapter{Exercise 5: The Structure Of A C Program}

You know how to use \ident{printf} and have a couple basic tools
at your disposal, so let's break down a simple C program line-by-line
so you know how one is structured.  In this program you're going to
type in a few more things that you're unfamiliar with, and I'm going
to lightly break them down.  Then in the next few exercises we're 
going to work with these concepts.

\begin{code}{ex5.c}
<< d['code/ex5.c|pyg|l'] >>
\end{code}

Type this code in, make it run, and make sure you get \emph{no Valgrind
errors}.  You probably won't but get in the habit of checking it. 

\section{What You Should See}

This has pretty boring output, but the point of this exercise is
to analyze the code:

\begin{code}{ex5 output}
\begin{lstlisting}
<< d['code/ex5.out|dexy'] >>
\end{lstlisting}
\end{code}


\section{Breaking It Down}

There's a few features of the C language in this code that you 
might have only slightly figured out while you were typing code.
Let's break this down line-by-line quickly, and then we can do
exercises to understand each part better:

\begin{description}
\item[ex5.c:1] An \ident{include} and it is way to import the contents
    of one file into this source file.  C has a convention of using 
    \file{.h} extensions for "header" files, which then contain lists
    of functions you want to use in your program.
\item[ex5.c:3] This is a multi-line \ident{comment} and you could put as many
    lines of text between the \verb|/*| and closing \verb|*/| characters.
\item[ex5.c:4] A more complex version of the \ident{main function} you've
    been using blindly so far.  How C programs work is the operating system
    loads your program, and then runs the function named \ident{main}.
    For the function to be totally complete it needs to return an \ident{int}
    and take two parameters, an \ident{int} for the argument count, and
    an array of \ident{char *} strings for the arguments.  Did that
    just fly over your head?  Do not worry, we'll cover this soon.
\item[ex5.c:5] To start the body of any function you write a \verb|{| character
    that indicates the beginning of a "block".  In Python you just
    did a \verb|:| and indented. In other languages you might have a
    \ident{begin} or \ident{do} word to start.
\item[ex5.c:6] A variable declaration and assignment at the same time.
    This is how you create a variable, with the syntax \verb|type name = value;|.
    In C statements (except for logic) end in a \verb|';'| (semicolon) character.
\item[ex5.c:8] Another kind of comment, and it works like Python or
    Ruby comments where it starts at the \verb|//| and goes until the end of
    the line.
\item[ex5.c:9] A call to your old friend \ident{printf}.  Like in many languages
    function calls work with the syntax \verb|name(arg1, arg2);| and can have
    no arguments, or any number.  The \ident{printf} function is actually
    kind of weird and can take multiple arguments. We'll see that later.
\item[ex5.c:11] A return from the main function, which gives the OS your exit
    value.  You may not be familiar with how Unix software uses return
    codes, so we'll cover that as well.
\item[ex5.c:12] Finally, we end the main function with a closing brace \verb|}|
    character and that's the end of the program.
\end{description}

There's a lot of information in this break-down, so study it line-by-line
and make sure you at least have a little grasp of what's going on.  You
won't know everything, but you can probably guess before we continue.

\section{Extra Credit}

\begin{enumerate}
\item For each line, write out the symbols you don't understand and
    see if you can guess what they mean.  Write a little chart on 
    paper with your guess that you can use to check later and see
    if you get it right.
\item Go back to the source code from the previous exercises and
    do a similar break-down to see if you're getting it.  Write down
    what you don't know and can't explain to yourself.
\end{enumerate}


