\chapter{Exercise 6: Types Of Variables}

You should be getting a grasp of how a simple C program is structured,
so let's do the next simplest thing which is making some variables 
of different types:

\begin{code}{ex6.c}
<< d['code/ex6.c|pyg|l'] >>
\end{code}

In this program we're declaring variables of different types 
and then printing them with different \ident{printf} format
strings.

\section{What You Should See}

Your output should look like mine, and you can start to see how
the format strings for C are similar to Python and other languages.
They've been around for a long time.

\begin{code}{ex6 output}
\begin{lstlisting}
<< d['code/ex6.out|dexy'] >>
\end{lstlisting}
\end{code}

What you can see is we have a set of "types", which are ways of
telling the C compiler what each variable should represent, and then
format strings to match different types.  Here's the breakdown
of how they match up:

\begin{description}
\item[Integers] You declare Integers with the \ident{int} keyword, and 
    print them with \verb|%d|.
\item[Floating Point] Declared with \ident{float} or \ident{double} depending
    on how big they need to be (double is bigger), and printed with 
    \verb|%f|.
\item[Character] Declared with \ident{char}, written with a \verb|'| (single-quote)
    character around the char, and then printed with \verb|%c|.
\item[String (Array of Characters)] Declared with \ident[char name\[\]], 
    written with \verb|"| characters, and printed with \verb|%s|.
\end{description}

You'll notice that C makes a distinction between single-quote for \ident{char}
and double-quote for \ident{char[]} or strings.  A short-hand way to write
out a type, mostly because sometimes the variable name is in the middle, is
to do like \ident{char[]} when we mean \ident{char SOMENAME[]}.

\section{How To Break It}

You can easily break this program by passing the wrong thing to the 
printf statements.  For example, if you take the line that prints
my name, but put the \ident{initial} variable before the \ident{first\_name}
in the arguments, you'll get a bug.  Make that change and the compiler will
yell at you, then when you run it you *might* get a "Segmentation fault"
like I did:

\begin{code}{ex6 explosion}
\begin{lstlisting}
<< d['code/ex6.bad.out|dexy'] >>
\end{lstlisting}
\end{code}

Run this change under Valgrind too to see what it tells you about
the error "Invalid read of size 1".

\section{Extra Credit}

\begin{enumerate}
\item Come up with other ways to break this C code by changing the
    \ident{printf}, then fix them.
\item Go search for "printf formats" and try using a few of the
    more exotic ones.
\item Research how many different ways you can write a number. Try
    octal, hexadecimal, and others you can find.
\item Try printing an empty string that's just \verb|""|.
\end{enumerate}

