\chapter{Exercise 14: Writing And Using Functions}

Until now you've just used functions that are part of the
\file{stdio.h} header file.  In this exercise you will write
some functions and use some other functions.

\begin{code}{ex14.c}
<< d['code/ex14.c|pyg|l'] >>
\end{code}

In this example you're creating functions to print out the
characters and ASCII codes for any that are "alpha" or "blanks".
Here's the breakdown:

\begin{description}
\item[ex14.c:2] Include a new header file so we can gain access to
    \func{isalpha} and \func{isblank}.
\item[ex14.c:5-6] Tell C that you will be using some functions later
    in your program, without having to actually define them.
    This is a "forward declaration" and it solves the chicken-and-egg
    problem of needing to use a function before you've defined it.
\item[ex14.c:8-15] Define the \func{print\_arguments} which knows how
    to print the same array of strings that \func{main} typically
    gets.
\item[ex14.c:17-30] Define the next function \func{print\_letters} that 
    is called \emph{by} \func{print\_arguments} and knows
    how to print each of the characters and their codes.
\item[ex14.c:32-35] Define \func{can\_print\_int} which simply returns
    the truth value (0 or 1) of \verb,isalpha(ch) || isblank(ch),
    back to its caller \func{print\_letters}.
\item[ex14.c:38-42] Finally \func{main} simply calls \func{print\_arguments}
    to make the whole chain of function calls go.
\end{description}

I shouldn't have to describe what's in each function because it's all
things you've ran into before.  What you should be able to see though
is that I've simply defined functions the same way you've been defining
\func{main}.  The only difference is you have to help C out by telling
it ahead of time if you're going to use functions it hasn't encountered
yet in the file.  That's what the "forward declarations" at the top do.


\section{What You Should See}

To play with this program you just feed it different command line 
arguments, which get passed through your functions.  Here's me
playing with it to demonstrate:

\begin{code}{ex14 output}
\begin{lstlisting}
<< d['code/ex14.out|dexy'] >>
\end{lstlisting}
\end{code}

The \func{isalpha} and \func{isblank} do all the work of figuring
out if the given character is a letter or a blank.  When I do the
last run it prints everything but the '3' character, since that 
is a digit.

\section{How To Break It}

There's two different kinds of "breaking" in this program:

\begin{enumerate}
\item Confuse the compiler by removing the forward declarations
    so it complains about \func{can\_print\_it} and \func{print\_letters}.
\item When you call \func{print\_arguments} inside \func{main} try
    adding 1 to \ident{argc} so that it goes past the end of the
    \ident{argv} array.
\end{enumerate}


\section{Extra Credit}

\begin{enumerate}
\item Rework these functions so that you have fewer functions.  For example,
    do you really need \func{can\_print\_it}?
\item Have \func{print\_arguments} figure how long each argument string
    is using the \func{strlen} function, and then pass that length
    to \func{print\_letters}.  Then, rewrite \func{print\_letters} 
    so it only processes this fixed length and doesn't rely on the
    \verb|'\0'| terminator.
\item Use \program{man} to lookup information on \func{isalpha}
    and \func{isblank}.  Use the other similar functions to
    print out only digits or other characters.
\item Go read about how different people like to format their
    functions.  Never use the "K&R syntax" as it's antiquated and
    confusing, but understand what it's doing in case you run
    into someone who likes it.
\end{enumerate}

