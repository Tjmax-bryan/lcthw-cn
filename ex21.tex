\chapter{Exercise 21: Advanced Data Types And Flow Control}

This exercise will be a complete compendium of the available C data types and
flow control structures you can use.  It will work as a reference to complete
your knowledge, and won't have any code for you to enter.  I'll have you
memorize some of the information by creating flash cards so you can get 
the important concepts solid in your mind.

For this exercise to be useful, you should spend at least a week hammering 
the content and filling out all the element I have missing here.  You'll be
writing out what each one means, and then writing a program to confirm
what you've researched.


\section{Available Data Types}

\begin{description}
\item[int] Stores a regular integer, defaulting to 32 bits in size.
\item[double] Holds a large floating point number.
\item[float] Holds a smaller floating point number.
\item[char] Holds a single 1 byte character.
\item[void] Indicates "no type" and used to say a function returns
    nothing, or a pointer has no type as in \verb|void *thing|.
\item[enum] Enumerated types, work as integers, convert to integers,
    but give you symbolic names for sets.  Some compilers will warn
    you when you don't cover all elements of an enum in \ident{switch-statements}.
\end{description}

\subsection{Type Modifiers}

\begin{description}
\item[unsigned] Changes the type so that it does not have negative numbers,
    giving you a larger upper bound but nothing lower than 0.
\item[signed] Gives you negative and positive numbers, but halves your 
    upper bound in exchange for the same lower bound negative.
\item[long] Uses a larger storage for the type so that it can hold
    bigger numbers, usually doubling the current size.
\item[short] Uses smaller storage for the type so it stores less, but
    takes half the space.
\end{description}


\subsection{Type Qualifiers}

\begin{description}
\item[const] Indicates the variable won't change after being initialized.
\item[volatile] Indicates that all bets are off, and the compiler should leave
    this alone and try not to do any fancy optimizations to it.  You usually
    only need this if you're doing really weird stuff to your variables.
\item[register] Forces the compiler to keep this variable in a register, and
    the compiler can just ignore you.  These days compilers are better at 
    figuring out where to put variables, so only use this if you actually
    can measure it improving the speed.
\end{description}


\subsection{Type Conversion}

C uses a sort of "stepped type promotion" mechanism, where it looks at two operands on
either side of an expression, and promotes the smaller side to match the larger side
before doing the operation.  If one side of an expression is on this list, then the 
other side is converted to that type before the operation is done, and this goes
in this order:

\begin{enumerate}
\item long double
\item double 
\item float
\item int (but only \ident{char} and \ident{short int});
\item long
\end{enumerate}

If you find yourself trying to figure out how your conversions are working in
an expression, then don't leave it to the compiler.  Use explicit casting
operations to make it exactly what you want.  For example, if you have:

\verb|long + char - int * double|

Rather than trying to figure out if it will be converted to double correctly,
just use casts:

\verb|(double)long - (double)char - (double)int * double|

Putting the type you want in parenthesis before the variable name is how you
force it into the type you really need.  The important thing though is 
\emph{always promote up, not down}.  Don't cast \ident{long} into \ident{char}
unless you know what you're doing.

\subsection{Type Sizes}

The \file{stdint.h} defines both a set of \ident{typdefs} for exact sized integer
types, as well as a set of macros for the sizes of all the types.  This is easier
to work with than the older \file{limits.h} since it is consistent.  The types
defined are:

\begin{description}
\item[int8\_t] 8 bit signed integer.
\item[uint8\_t] 8 bit unsigned integer.
\item[int16\_t] 16 bit signed integer.
\item[uint16\_t] 16 bit unsigned integer.
\item[int32\_t] 32 bit signed integer.
\item[uint32\_t] 32 bit unsigned integer.
\item[int64\_t] 64 bit signed integer.
\item[uint64\_t] 64 bit unsigned integer.
\end{description}

The pattern here is of the form (u)int(BITS)\_t where a \emph{u} is put in front to indicate
"unsigned", then \emph{BITS} is a number for the number of bits.  This pattern is then repeated
for macros that return the maximum values of these types:

\begin{description}
\item[INT\emph{N}\_MAX] Maximum positive number of the signed integer of bits \emph{N}.
\item[INT\emph{N}\_MIN] Minimum negative number of signed integer of bits \emph{N}.
\item[UINT\emph{N}\_MAX] Maximum positive number of unsigned integer of bits \emph{N}. Since it's unsigned the minimum is 0 and can't have a negative value.
\end{description}

There are also macros in \file{stdint.h} for sizes of the \ident{size\_t} type, integers large enough to hold pointers,
and other handy size defining macros.  Compilers have to at least have these, and then they can allow
other larger types.

Here is a full list should be in \file{stdint.h}:

\begin{description}
\item[int\_least\emph{N}\_t] holds at least \emph{N} bits.
\item[uint\_least\emph{N}\_t] holds at least \emph{N} bits unsigned.
\item[INT\_LEAST\emph{N}\_MAX] max value of the matching least\emph{N} type.
\item[INT\_LEAST\emph{N}\_MIN] min value of the matching least\emph{N} type.
\item[UINT\_LEAST\emph{N}\_MAX] unsigned maximum of the matching \emph{N} type.
\item[int\_fast\emph{N}\_t] similar to \ident{int\_least\emph{N}\_t} but asking for the "fastest" with at least that precision.
\item[uint\_fast\emph{N}\_t] unsigned fastest least integer.
\item[INT\_FAST\emph{N}\_MAX] max value of the matching fastest\emph{N} type.
\item[INT\_FAST\emph{N}\_MIN] min value of the matching fastest\emph{N} type.
\item[UINT\_FAST\emph{N}\_MAX] unsigned max value of the matching fastest\emph{N} type.
\item[intptr\_t] a \emph{signed} integer large enough to hold a pointer.
\item[uintptr\_t] an \emph{unsigned} integer large enough to hold a pointer.
\item[INTPTR\_MAX] max value of a \ident{intptr\_t}.
\item[INTPTR\_MIN] min value of a \ident{intptr\_t}.
\item[UINTPTR\_MAX] unsigned max value of a \ident{uintptr\_t}.
\item[intmax\_t] biggest number possible on that system.
\item[uintmax\_t] biggest unsigned number possible.
\item[INTMAX\_MAX] largest value for the biggest signed number.
\item[INTMAX\_MIN] smallest value for the biggest signed number.
\item[UINTMAX\_MAX] largest value for the biggest unsigned number.
\item[PTRDIFF\_MIN] minimum value of \ident{ptrdiff\_t}.
\item[PTRDIFF\_MAX] maximum value of \ident{ptrdiff\_t}.
\item[SIZE\_MAX] maximum of a \ident{size\_t}.
\end{description}

\section{Available Operators}

This is a comprehensive list of all the operators you have in the C language.
In this list, I'm indicating the following:

\begin{description}
\item[(binary)] The operator has a left and right: \verb|X + Y|.
\item[(unary)] The operator is on its own: \verb|-X|.
\item[(prefix)] The operator comes before the variable: \verb|++X|.
\item[(postfix)] Usually the same as the \ident{(prefix)} version, but placing it
    after gives it a different meaning: \verb|X++|.
\item[(ternary)] There's only one of these, so it's actually called the
    ternary but it means "three operands": \verb|X ? Y : Z|.
\end{description}


\subsection{Math Operators}

These are your basic math operations, plus I put \verb|()| in
with these since it calls a function and is close to a "math"
operation.

\begin{description}
\item[()] Function call.
\item[* (binary)] multiply.
\item[/] divide.
\item[+ (binary)] addition.
\item[+ (unary)] positive number.
\item[++ (postfix)] increment, then read.
\item[++ (prefix)] read, then increment.
\item[$--$ (postfix)] increment, then read.
\item[$--$ (prefix)] read, then increment.
\item[- (binary)] subtract.
\item[- (unary)] negative number.
\end{description}

\subsection{Data Operators}

These are used to access data in different ways and forms.

\begin{description}
\item[-\textgreater{}] struct pointer access.
\item[.] struct value access.
\item{[]} Array index.
\item[sizeof] size of a type or variable.
\item[\& (unary)] Address of.
\item[* (unary)] Value of.
\end{description}

\subsection{Logic Operators}

These handle testing equality and inequality of variables.

\begin{description}
\item[!=] does not equal.
\item[\textless{}] less than.
\item[\textless{}=] less than or equal.
\item[==] equal (not assignment).
\item[\textgreater{}] greater than.
\item[\textgreater{}=] greater than or equal.
\end{description}

\subsection{Bit Operators}

These are more advanced and for shifting and modifying the 
raw bits in integers.

\begin{description}
\item[\& (binary)] Bitwise and.
\item[\textless{}\textless{}] Shift left.
\item[\textgreater{}\textgreater{}] Shift right.
\item[\^{}] bitwise xor (exclusive or).
\item[$\vert$] bitwise or.
\item[\textasciitilde{}] compliment (flips all the bits).
\end{description}


\subsection{Boolean Operators}

Used in truth testing.  Study the ternary operator carefully, it is very handy.

\begin{description}
\item[!] not.
\item[\&\&] and.
\item[$\vert\vert$] or.
\item[?:] Ternary truth test, read \verb|X ? Y : Z| as "if X then Y else Z".
\end{description}

\subsection{Assignment Operators}

Compound assignment operators that assign a value, and/or perform
an operation at the same time.  Most of the above operations
can also be combined into a compound assignment operator.

\begin{description}
\item[=] assign.
\item[\%=] modulus assign.
\item[\&=] bitwise and assign.
\item[*=] multiply assign.
\item[+=] plus assign.
\item[-=] minus assign.
\item[/=] divid assign.
\item[\textless{}\textless{}=] shift left, assign.
\item[\textgreater{}\textgreater{}=] shift right, assign.
\item[\^{}=] bitwise xor, assign.
\item[$\vert$=] bitwise or, assign.
\end{description}

\section{Available Control Structures}

There's a few control structures you haven't encountered yet:

\begin{description}
\item[do-while] \verb|do { ... } while(X);| First does the code in the block, then
    tests the \ident{X} expression before exiting.
\item[break] Put this in a loop, and it breaks out ending it early.
\item[continue] Stops the body of a loop and jumps to the test so it can continue.
\item[goto] Jumps to a spot in the code where you've placed a \verb|label:|, and
    you've been using this in the \file{dbg.h} macros to go to the \verb|error:|
    label.
\end{description}


\subsection{Extra Credit}

\begin{enumerate}
\item Read \file{stdint.h} or a description of it and write out all the possible
    available size identifiers.
\item Go through each item here and write out what it does in code.  Research it so you
     know you got it right by looking it up online.
\item Get this information solid as well by making flash cards and spending 15 
    minutes a day memorizing it.
\item Create a program that prints out examples of each type and confirm that your
    research is right.
\end{enumerate}

