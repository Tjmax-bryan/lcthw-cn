\chapter{习题 0: 准备工作}

本章你将学会架设 C 语言编程的系统环境。如果你使用的是 Linux 或者 Mac OSX,那么有个好消息
可以告诉你,你的系统本来就是\emph{为} C 语言编程设计的。C 语言的发明人也曾是 Unix 操作
系统的创作者,而 Linux 和 OSX 都是基于 Unix 的操作系统。所以,整个架设过程会非常简单。

对于 Windows 的用户,我就只有坏消息了: 在 Windows 下学习 C 语言是一件痛苦的事情。在 
Windows 下写 C 语言代码不是问题,问题是 Windows 下所有的库、函数、以及工具和 Unix 下的
C 语言环境比起来就是有那么一点“跑偏”的感觉。恐怕这也是你不得不接受的事实了。

你也别被这条突如其来的坏消息吓到。我不是说要完全避开 Windows,我的意思是说,如果你想最不
费力地学习C语言,那么你最好还是从 Unix 下手。了解一点 Unix 还有另外一个好处,那就是你可以
学到一些 C语言的惯用技巧,从而扩展你的编程技术。

这也意味着你将会用到\emph{命令行}。没错,我是说命令行。你需要在命令行输入命令。不过别害怕,
我会告诉你该输入什么命令,以及执行命令会有什么样的结果,这样你同时也会学到不少让你大开眼界
的技能。

\section{Linux}

对于大部分 Linux 系统来说,你只需要安装若干软件包就可以了。在基于 Debian 的系统(例如 
Ubuntu)上面,你只要使用下面的命令安装即可:

\begin{code}{在 Ubuntu 上面安装需求软件包}
\begin{lstlisting}
$ sudo aptitude install build-essential
\end{lstlisting}
\end{code}

以上是一个命令行的例子,所以你要先找到系统里的“命令行终端(terminal)”并且把它运行起来,
才能执行上述的命令。你会看到一个类似上面提到的 '\$' 的提示界面,然后键入上述命令。\emph{
'\$' 这个符号是无需键入的,只要把后面的内容键入即可。}

Here's how you would install the same setup on an RPM based Linux
like Fedora:

\begin{code}{在 Fedora 上面安装需求软件包}
\begin{lstlisting}
$ su -c yum groupinstall development-tools
\end{lstlisting}
\end{code}

Once you've run that, you should be able to do the first Exercise in
this book and it'll work.  If not then let me know.


\section{Mac OSX}

On Mac OSX the install is even easier.  First, you'll need to either 
download the latest \ident{XCode} from Apple, or find your install
DVD and install it from there.  The download will be massive and could
take forever, so I recommend installing from the DVD.  Also, search
online for "installing xcode" for instructions on how to do it.

Once you're done installing XCode, and probably restarting your computer
if it didn't make you do that, you can go find your Terminal program
and get it put into your Dock.  You'll be using Terminal a lot in
the book, so it's good to put it in a handy location.


\section{Windows}

For Windows users I'll show you how to get a basic Ubuntu Linux system up and
running in a virtual machine so that you can still do all of my exercises, but
avoid all the painful Linux installation problems.

... have to figure this one out.


\section{Text Editor}

The choice of text editor for a programmer is a tough one.  For beginners
I tell them to just use \href{http://projects.gnome.org/gedit/}{Gedit} since
it's simple and works for code.  However, it doesn't work in certain
internationalized situations, and chances are you already have a favorite
text editor if you've been programming for a while.

With this in mind, I want you to try out a few of the standard programmer
text editors for your platform and then stick with the one that you like
best.  If you've been using GEdit and like it then stick with it.  If you
want to try something different, then try it out real quick and pick one.

The most important thing is \emph{do not get stuck picking the perfect editor}.
Text editors all just kind of suck in odd ways.  Just pick one, stick with it,
and if you find something else you like try it out.  Don't spend days
on end configuring it and making it perfect.

Some text editors to try out are:

\begin{enumerate}
\item \href{http://projects.gnome.org/gedit/}{Gedit} on Linux and OSX.
\item \href{http://www.barebones.com/products/textwrangler/}{TextWrangler} on OSX.
\item \href{http://www.nano-editor.org/}{Nano} which runs in Terminal and works nearly everywhere.
\item \href{http://www.gnu.org/software/emacs/}{Emacs} and \href{http://emacsformacosx.com/}{Emacs for OSX}.  Be prepared to do some learning though.
\item \href{http://www.vim.org/}{Vim} and \href{http://code.google.com/p/macvim/}{MacVim}
\end{enumerate}

There is probably a different editor for every person out there, but these are
just a few of the free ones that I know work.  Try a few out, and maybe some
commercial ones until you find one that you like.

\subsection{WARNING: Do Not Use An IDE}

An IDE, or "Integrated Development Environment" will turn you stupid.  They are
the worst tools if you want to be a good programmer because they hide what's
going on from you, and your job is to know what's going on.  They are useful
if you're trying to get something done and the platform is designed around 
a particular IDE, but for learning to code C (and many other languages) they
are pointless.

\begin{aside}{IDE和吉他谱}

玩过吉他的人都知道吉他谱是什么东西,不过还是让我给其他没玩过吉他的人解释一下吧。有一种约定俗成
的音乐记谱方式叫做“线谱”,这是一种普遍的,古老的,通用的记录如何演奏乐器的方法。线谱很大程度上
是为钢琴和作曲家而生,所以如果你弹钢琴的话,线谱是很容易使用的。

然而吉他这种乐器有些古怪,它并不适合这种记谱方式,所以演奏吉他的人使用了一种另类的记谱方式,称
作“吉他谱(tablature)”。吉他谱告诉你的不是要演奏的音调,而是你在某一时刻要弹的指位和琴弦。你可
以在不了解任何曲调的情况下学会弹奏一首曲子,很多人也是这么去学的。然而如果你想从中读出你弹奏的
\emph{曲调},吉他谱就没什么用处了。

传统的记谱方式也许比吉他谱难学,但它可以告诉你如何演奏\emph{音乐},而不仅仅是如何弹吉他。拿着
一份线谱,我可以走到一架钢琴前面弹出同样的一首歌曲,我可以用贝司把它弹出来,我还可以把它输入到
计算机中重新设计整份乐谱。然而拿着吉他谱,我就只能用它弹弹吉他。

IDE和吉他谱类似。毫无疑问你可以使用IDE快速地写出代码,但你只能在一个固定的平台上使用一种特定的
语言。这也是公司企业喜欢兜售这些东西给你的原因。他们知道你是个懒人,而IDE只在他们的平台上面工作,
就这样,由于你的懒惰,他们就把你禁锢在他们的平台上了。

打破这个循环的方法也不是没有,你需要卧薪尝胆,最终学会如何不使用IDE进行编程。简单的文本编辑器,
或者像Vim和Emacs这样的程序员编辑器,会让代码真正成为你的工作对象。比起使用IDE来这样会更难一些,
不过最终的结果就是你可以应对\emph{任何}代码,不管它在什么样的计算机平台上,不管它使用的是什么
语言,而且你懂它的深层原理。

\end{aside}


