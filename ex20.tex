\chapter{Exercise 20: Zed's Awesome Debug Macros}

There is a constant problem in C that you have been dancing around but which I
am going to solve in this exercise using a set of macros I developed.  You can
thank me later when you realize how insanely awesome these macros are.  Right
now you won't realize how awesome they are, so you'll just have to use them and
then you can walk up to me one day and say, "Zed, those Debug Macros were the
bomb. I owe you my first born child because you saved me a decade of heartache
and prevented me from killing myself more than once. Thank you good sir, here's
a million dollars and the original Snakehead Telecaster prototype signed by Leo
Fender."

Yes, they are that awesome.

\section{The C Error Handling Problem}

In almost every programming language handling errors is a difficult activity.
There's entire programming languages that try as hard as they can to avoid even
the concept of an error.  Other languages invent complex control structures
like exceptions to pass error conditions around.  The problem exists mostly
because programmers assume errors don't happen and this optimism infects the
type of languages they use and create.

C tackles the problem by returning error codes and setting a global
\ident{errno} value that you check.  This makes for complex code that
simply exists to check if something you did had an error.  As you 
write more and more C code you'll write code with the pattern:

\begin{enumerate}
\item Call a function.
\item If the return value is an error (must look that up each time too).
\item Then cleanup all the resource created so far.
\item and print out an error message that hopefully helps.
\end{enumerate}

This means for every function call (and yes, \emph{every} function)
you are potentially writing 3-4 more lines just to make sure it worked.
That doesn't include the problem of cleaning up all of the junk you've
built to that point.  If you have 10 different structures, 3 files, and
a database connection, when you get an error then you
would have 14 more lines.

In the past this wasn't a problem because C programs did what you've
been doing when there's an error: die.  No point in bothering with cleanup
when the OS will do it for you.  Today though many C programs need to run
for weeks, months, or years and handle errors from many different sources
gracefully.  You can't just have your webserver die at the slightest 
touch, and you definitely can't have a library you've written nuke a
the program its used in.  That's just rude.

Other languages solve this problem with exceptions, but those have problems
in C (and in other languages too).  In C you only have one return value,
but exceptions are an entire stack based return system with arbitrary
values.  Trying to marshal exceptions up the stack in C is difficult, and
no other libraries will understand it.

\section{The Debug Macros}

The solution I've been using for years is a small set of "debug macros"
that implement a basic debugging and error handling system for C.  This
system is easy to understand, works with every library, and makes C code
more solid and clearer.

It does this by adopting the convention that whenever there's an error, your
function will jump to an "error:" part of the function that knows how to
cleanup everything and return an error code.  You use a macro called
\func{check} to check return codes, print an error message, and then
jump to the cleanup section.  You combine that with a set of logging
functions for printing out useful debug messages.

I'll now show you the entire contents of the most awesome set of brilliance
you've ever seen:

\begin{code}{dbg.h}
<< d['code/dbg.h|pyg|l'] >>
\end{code}

Yes, that's it, and here's what every line does:

\begin{description}
\item[dbg.h:1-2] The usual defense against accidentally including the file
    twice, which you saw in the last exercise.
\item[dbg.h:4-6] Includes for the functions that these macros need.
\item[dbg.h:8] The start of a \ident{\#ifdef} which let's you recompile your
    program so that all the debug log messages are removed.
\item[dbg.h:9] If you compile with \ident{NDEBUG} defined, then "no debug" messages
    will remain.  You can see in this case the \ident{\#define debug()} is just
    replaced with nothing (the right side is empty).
\item[dbg.h:10] The matching \ident{\#else} for the above \ident{\#ifdef}.
\item[dbg.h:11] The alternative \ident{\#define debug} that translates any use
    of \verb|debug("format", arg1, arg2)| into an \func{fprintf} call 
    to \ident{stderr}.  Many C programmers don't know, but you can create
    macros that actually work like \func{printf} and take variable arguments.
    Some C compilers (actually cpp) don't support this, but the ones that matter do.
    The magic here is the use of \verb|##__VA_ARGS__| which says 
    "put whatever they had for extra arguments (...) here".  Also notice 
    the use of \ident{\_\_FILE\_\_} and \ident{\_\_LINE\_\_} to get the
    current file:line for the debug message.  \emph{Very} helpful.
\item[dbg.h:12] The end of the \ident{\#ifdef}.
\item[dbg.h:14] The \func{clean\_errno} macro that's used in the others to get a
    safe readable version of \ident{errno}.  That strange syntax in the
    middle is a "ternary operator" and you'll learn what it does later.
\item[dbg.h:16-20] The \func{log\_err}, \func{log\_warn}, and \func{log\_info},
    macros for logging messages meant for the end user.  Works
    like \func{debug} but can't be compiled out.
\item[dbg.h:22] The best macro ever, \func{check} will make sure the condition
    \ident{A} is true, and if not logs the error \ident{M} (with variable
    arguments for \func{log\_err}), then jumps to the function's \ident{error:}
    for cleanup.
\item[dbg.h:24] The 2nd best macro ever, \func{sentinel} is placed in any part of a
    function that shouldn't run, and if it does prints an error message then
    jumps to the \ident{error:} label.  You put this in \ident{if-statements}
    and \ident{switch-statements} to catch conditions that shouldn't happen,
    like the \ident{default:}.
\item[dbg.h:26] A short-hand macro \func{check\_mem} that makes sure a pointer is 
    valid, and if it isn't reports it as an error with "Out of memory."

\item[dbg.h:28] An alternative macro \func{check\_debug} that still checks and
    handles an error, but if the error is common then you don't want to bother
    reporting it.  In this one it will use \func{debug} instead of \func{log\_err}
    to report the message, so when you define \ident{NDEBUG} the check still
    happens, the error jump goes off, but the message isn't printed.
\end{description}

\section{Using dbg.h}

Here's an example of using all of \file{dbg.h} in a small program.
This doesn't actually do anything but demonstrate how to use each
macro, but we'll be using these macros in all of the programs we
write from now on, so be sure to understand how to use them.

\begin{code}{ex20.c}
<< d['code/ex20.c|pyg|l'] >>
\end{code}

Pay attention to how \func{check} is used, and how when it is
\ident{false} it will jump to the \ident{error:} label to do a cleanup.
The way to read those lines is, "check that A is true and if not say M
and jump out."


\section{How The C Pre-Processor Works}

It's now time for you to get a small introduction to the CPP so that you
know how these macros actually work.  To do this, I'm going to break down
the most complex macro from \file{dbg.h} and have you run \program{cpp} so
you can see what it's actually doing.

More to come...

\section{What You Should See}

When you run this, give it some bogus first parameter and you should see
this:

\begin{code}{ex20 output}
\begin{lstlisting}
<< d['code/ex20.out|dexy'] >>
\end{lstlisting}
\end{code}

See how it reports the exact line number where the \func{check} failed?
That's going to save you hours of debugging later.  See also how it 
prints the error message for you when \ident{errno} is set? Again,
that will save you hours of debugging.

\section{Extra Credit}

\begin{enumerate}
\item Put \verb|#define NDEBUG| at the top of the file and check that all
    the debug messages go away.
\item Undo that line, and add \verb|-DNDEBUG| to \ident{CFLAGS} at the
    top of the \file{Makefile} then recompile to see the same thing.
\item Modify the logging so that it include the function name as well
    as the file:line.
\end{enumerate}

