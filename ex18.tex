\chapter{Exercise 18: Pointers To Functions}

Functions in C are actually just pointers to a spot in the
program where some code exists.  Just like you've been creating
pointers to structs, strings, and arrays, you can point a
pointer at a function too.  The main use for this is to pass
"callbacks" to other functions, or to simulate classes and
objects.  In this exercise we'll do some callbacks, and in
the next one we'll make a simple object system.

The format of a function pointer goes like this:

\verb|int (*POINTER_NAME)(int a, int b)|

A way to remember how to write one is to do this:

\begin{enumerate}
\item Write a normal function declaration: \verb|int callme(int a, int b)|
\item Wrap function name with pointer syntax: \verb|int (*callme)(int a, int b)|
\item Change the name to the pointer name: \verb|int (*compare_cb)(int a, int b)|
\end{enumerate}

The key thing to remember is, when you're done with this, the \emph{variable}
name for the pointer is called \emph{compare\_cb} and then you use it
just like it's a function.  This is similar to how pointers to arrays can
be used just like the arrays they point to.  Pointers to functions can be
used like the functions they point to but with a different name.

\begin{code}{Using A Raw Function Pointer}
\begin{lstlisting}
    int (*tester)(int a, int b) = sorted_order;
    printf("TEST: %d is same as %d\n", tester(2, 3), sorted_order(2, 3)); 
\end{lstlisting}
\end{code}

This will work even if the function pointer returns a pointer to something:

\begin{enumerate}
\item Write it: \verb|char *make_coolness(int awesome_levels)|
\item Wrap it: \verb|char *(*make_coolness)(int awesome_levels)|
\item Rename it: \verb|char *(*coolness_cb)(int awesome_levels)|
\end{enumerate}

The next problem to solve with using function pointers is that it's hard to
give them as paramters to a function, like when you want to pass the function
callback to another function.  The solution to this is to use \ident{typedef}
which is a C keyword for making new names for other more complex types.
The only thing you need to do is put \ident{typedef} before the same 
function pointer syntax, and then after that you can use the name like
it's a type.  I demonstrate this in the following exercise code:

\begin{code}{ex18.c}
<< d['code/ex18.c|pyg|l'] >>
\end{code}

In this program you're creating a dynamic sorting algorithm that can sort
an array of integers using a comparison callback.  Here's the breakdown
of this program so you can clearly understand it:

\begin{description}
\item[ex18.c:1-6] The usual includes needed for all the functions we call.
\item[ex18.c:7-17] This is the \func{die} function from the previous exercise
    which I'll use to do error checking.
\item[ex18.c:21] This is where the \ident{typedef} is used, and later I use 
    \ident{compare\_cb} like it's a type similar to \ident{int} or 
    \ident{char} in \func{bubble\_sort} and \func{test\_sorting}.
\item[ex18.c:27-49] A bubble sort implementation, which is a very inefficient way
    to sort some integers.  This function contains:
    \begin{description}
    \item[ex18.c:27] Here's where I use the \ident{typedef} for \ident{compare\_cb}
        as the last parameter \ident{cmp}.  This is now a function that will
        return a comparison between two integers for sorting.
    \item[ex18.c:29-34] The usual creation of variables on the stack, followed by
    a new array of integers on the heap using \func{malloc}.  Make sure you
    understand what \verb|count * sizeof(int)| is doing.
    \item[ex18.c:38] The outer-loop of the bubble sort.
    \item[ex18.c:39] The inner-loop of the bubble sort
    \item[ex18.c:40] Now I call the \func{cmp} callback just like it's a normal
        function, but instead of being the name of something we defined, 
        it's just a pointer to it.  This lets the caller pass in anything
        they want as long as it matches the "signature" of the \ident{compare\_cb}
        \ident{typedef}.
    \item[ex18.c:41-43] The actual swapping operation a bubble sort needs to do what it
        does.
    \item[ex18.c:48] Finally return the newly created and sorted result array \ident{target}.
    \end{description}
\item[ex18.c:51-68] Three different versions of the \ident{compare\_cb} function type,
    which needs to have the same definition as the \ident{typedef} we created.
    The C compiler will complain to you if you get this wrong and say the types don't
    match.
\item[ex18.c:74-87] This is a tester for the \func{bubble\_sort} function.  You can 
    see now how I'm also using \ident{compare\_cb} to then pass to \func{bubble\_sort}
    demonstrating how these can be passed around like any other pointers.
\item[ex18.c:90-103] A simple main function that sets up an array based on integers
    you pass on the command line, then calls the \func{test\_sorting} function.
\item[ex18.c:105-107] Finally, you get to see how the \ident{compare\_cb} function
    pointer \ident{typedef} is used.  I simply call \func{test\_sorting} but
    give it the name of \func{sorted\_order}, \func{reverse\_order}, and 
    \func{strange\_order} as the function to use.  The C compiler then finds
    the address of those functions, and makes it a pointer for 
    \func{test\_sorting} to use.  If you look at \func{test\_sorting} you'll
    see it then passes each of these to \func{bubble\_sort} but it actually
    has no idea what they do, only that they match the \ident{compare\_cb} 
    prototype and should work.
\item[ex18.c:109] Last thing we do is free up the array of numbers we made.
\end{description}


\section{What You Should See}

Running this program is simple, but try different combinations of numbers, and
try even non-numbers to see what it does.

\begin{code}{ex18 output}
\begin{lstlisting}
<< d['code/ex18.out|dexy'] >>
\end{lstlisting}
\end{code}


\section{How To Break It}

I'm going to have you do something kind of weird to break this.  These function
pointers are pointers like every other pointer, so they point at blocks of
memory.  C has this ability to take one pointer and convert it to another so
you can process the data in different ways.  It's usually not necessary, but
to show you how to hack your computer, I want you to add this at the end of
\func{test\_sorting}:

\begin{code}{Function Pointer Evil}
\begin{lstlisting}
    unsigned char *data = (unsigned char *)cmp;

    for(i = 0; i < 25; i++) {
        printf("%0x:", data[i]);
    }
    printf("\n");
\end{lstlisting}
\end{code}

This loop is sort of like converting your function to a string and then
printing out it's contents.  This won't break your program unless the CPU and
OS you're on has a problem with you doing this.  What you'll see is a string of
hexadecimal numbers after it prints the sorted array:

\begin{Verbatim}
55:48:89:e5:89:7d:fc:89:75:f8:8b:55:fc:8b:45:f8:29:d0:c9:c3:55:48:89:e5:89:
\end{Verbatim}

That should be the raw assembler byte code of the function itself, and you
should see they start the same, but then have different endings.  It's also
possible that this loop isn't getting all of the function or is getting too
much and stomping on another piece of the program.  Without more analysis
you wouldn't know.

\section{Extra Credit}

\begin{enumerate}
\item Get a hex editor and open up \program{ex18}, then find this sequence
    of hex digits that start a function to see if you can find the function
    in the raw program.
\item Find other random things in your hex editor and change them.  Rerun your
    program and see what happens.  Changing strings you find are the easiest
    things to change.
\item Pass in the wrong function for the \ident{compare\_cb} and see what
    the C compiler complains about.
\item Pass in NULL and watch your program seriously bite it.  Then run
    \program{Valgrind} and see what that reports.
\item Write another sorting algorithm, then change \func{test\_sorting} so
    that it takes \emph{both} an arbitrary sort function and the sort function's
    callback comparison.  Use it to test both of your algorithms.
\end{enumerate}


