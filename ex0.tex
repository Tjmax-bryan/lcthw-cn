\chapter{Exercise 0: The Setup}

In this chapter you get your system setup to do C programming.  The
good news for anyone using Linux or Mac OSX is that you are on a
system designed \emph{for} programming in C.  The authors of the
C language were also instrumental in the creation of the Unix operating
system, and both Linux and OSX are based on Unix.  In fact, the install
will be incredibly easy.

I have some bad news for users of Windows:  learning C on Windows
is painful.  You can write C code for Windows, that's not a problem.
The problem is all of the libraries, functions, and tools are just
a little "off" from everyone else in the C world.  C came from
Unix and is much easier on a Unix platform.  It's just a fact 
of life that you'll have to accept I'm afraid.

I wanted to get this bad news out right away so that you don't
panic.  I'm not saying to avoid Windows entirely.  I am however
saying that, if you want to have the easiest time learning C, then
it's time to bust out some Unix and get dirty.  This could also
be really good for you, since knowing a little bit of Unix will
also teach you some of the idioms of C programming and expand
your skills.

This also means that for everyone you'll be using the 
\emph{command line}.  Yep, I said it.  You've gotta get in there and
type commands at the computer.  Don't be afraid though because
I'll be telling you what to type and what it should look like,
so you'll actually be learning quite a few mind expanding 
skills at the same time.

\section{Linux}

On most Linux systems you just have to install a few packages.  For 
Debian based systems, like Ubuntu you should just have to install
a few things using these commands:

\begin{code}{Installing Requirements On Ubuntu}
\begin{lstlisting}
$ aptitude install build-essential
\end{lstlisting}
\end{code}

The above is an example of a command line prompt, so to get to where
you can run that, find your "Terminal" program and run it first.  Then
you'll get a shell prompt similar to the '\$' above and can type that
command into it.  \emph{Do not type the '\$', just the stuff after it.}

Once you've run that, you should be able to do the first Exercise in
this book and it'll work.  If not then let me know.


\section{Mac OSX}

On Mac OSX the install is even easier.  First, you'll need to either 
download the latest \ident{XCode} from Apple, or find your install
DVD and install it from there.  The download will be massive and could
take forever, so I recommend installing from the DVD.  Also, search
online for "installing xcode" for instructions on how to do it.

Once you're done installing XCode, and probably restarting your computer
if it didn't make you do that, you can go find your Terminal program
and get it put into your Dock.  You'll be using Terminal a lot in
the book, so it's good to put it in a handy location.


\section{Windows}

For Windows users I'll show you how to get a basic Ubuntu Linux system up and
running in a virtual machine so that you can still do all of my exercises, but
avoid all the painful Linux installation problems.

... have to figure this one out.


\section{Text Editor}

The choice of text editor for a programmer is a tough one.  For beginners
I tell them to just use \href{http://projects.gnome.org/gedit/}{Gedit} since
it's simple and works for code.  However, it doesn't work in certain
internationalized situations, and chances are you already have a favorite
text editor if you've been programming for a while.

With this in mind, I want you to try out a few of the standard programmer
text editors for your platform and then stick with the one that you like
best.  If you've been using GEdit and like it then stick with it.  If you
want to try something different, then try it out real quick and pick one.

The most important thing is \emph{do not get stuck picking the perfect editor}.
Text editors all just kind of suck in odd ways.  Just pick one, stick with it,
and if you find something else you like try it out.  Don't spend days
on end configuring it and making it perfect.

Some text editors to try out are:

\begin{enumerate}
\item \href{http://projects.gnome.org/gedit/}{Gedit} on Linux and OSX.
\item \href{http://www.barebones.com/products/textwrangler/}{TextWrangler} on OSX.
\item \href{http://www.nano-editor.org/}{Nano} which runs in Terminal and works nearly everywhere.
\item \href{http://www.gnu.org/software/emacs/}{Emacs} and \href{http://emacsformacosx.com/}{Emacs for OSX}.  Be prepared to do some learning though.
\item \href{http://www.vim.org/}{Vim} and \href{http://code.google.com/p/macvim/}{MacVim}
\end{enumerate}

There is probably a different editor for every person out there, but these are
just a few of the free ones that I know work.  Try a few out, and maybe some
commercial ones until you find one that you like.

\subsection{WARNING: Do Not Use And IDE}

An IDE, or "Integrated Development Environment" will turn you stupid.  They are
the worst tools if you want to be a good programmer because they hide what's
going on from you, and your job is to know what's going on.  They are useful
if you're trying to get something done and the platform is designed around 
a particular IDE, but for learning to code C (and many other languages) they
are pointless.

\begin{aside}{IDEs and Guitar Tablature}
If you've played guitar then you know what tablature is, but for everyone else
let me explain.  In music there's an established notation called the "staff notation".  It's a generic, very old, and universal way to write down what someone
should play on an instrument.  If you play piano this notation is fairly 
easy to use, since it was created mostly for piano and composers.

Guitar however is a weird instrument that doesn't really work with notation,
so guitarists have an alternative notation called "tablature".  What tablature
does is, rather than tell you the not to play, it tells you the fret
and string you should play at that time.  You could learn whole songs without
ever knowing about a single thing you're playing.  Many people do it this way,
but if you want to know \emph{what} you're playing, then tablature is pointless.

It may be easier, but traditional notation tells you how to play the \emph{music} rather than just how to play the guitar.  With traditional notation I can
walk over to a piano and play the same song.  I can play it on a bass.  I can
put it into a computer and design whole scores around it.  With tablature I can
just play it on a guitar.

IDEs are like tablature.  Sure, you can code pretty quickly, but you can only
code in that one language on that one platform.  This is why companies love 
selling them to you.  They know you're lazy, and since it only works on their
platform they've got you locked in because you are lazy.

The way you break the cycle is you suck it up and finally learn to code without
an IDE.  A plain editor, or a programmer's editor like Vim or Emacs, makes
you work with the code.  It's a little harder, but the end result is you can
work with \emph{any} code, on any computer, in any language, and you
know what's going on.
\end{aside}


