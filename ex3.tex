\chapter{Exercise 3: Formatted Printing}

Many programming languages use the C way of formatting output, so let's try it:

\begin{code}{ex3.c}
<< d['code/ex3.c|pyg|l'] >>
\end{code}

Once you have that, do the usual \shell{make ex3} to build it and run it.

This exercise has a whole lot going on in a small amount of code.  First you're
including another "header file" called \file{stdio.h}, then you're using a variable 
named \ident{age} and setting it to 10.  Then you're printing it out using the
function \ident{printf} and a format string similar to what you would use
in Python.

... explain this ...

\section{What You Should See}


\section{How To Break It}


\section{Extra Credit}



