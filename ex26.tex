\chapter{Exercise 26: Write A First Real Program}

You are at the half-way mark in the book, so you need to take a 
mid-term.  In this mid-term you're going to recreate a piece
of software I wrote specifically for this book called \program{devpkg}.
You'll then extend it in a few key ways and improve the code, most
importantly by writing some unit tests for it.

\section{What Is \program{devpkg}?}

\program{Devpkg} is a simple C program that installs other software.  I
made it specifically for this book as a way to teach you how a real 
software project is structured, and also how to reuse other people's
libraries.  It uses a portability library called \href{http://apr.apache.org/}{The
Apache Portable Runtime (APR)} that has many handy C functions which
work on tons of platforms, including Windows.

\begin{aside}{Do Not Cheat}
The source code to \program{devpkg} is already online in a repository
I created, so you could be a loser and just check it out.  I put it up
because it is actually useful for you to go check your work against
mine and see what mistakes you've made.  \emph{However}, resist the
urge to do this until you are truly really stuck.
\end{aside}

\subsection{What We Want To Make}

We want a tool that has three commands:

\begin{enumerate}
\item[devpkg -S] Sets up a new install on a computer.
\item[devpkg -I] Installs a piece of software from a URL.
\item[devpkg -L] Lists all the software that's been installed.
\end{enumerate}

We want \program{devpkg} to be able to take almost any URL, figure
out what kind of project it is, download it, install it, and register
that it downloaded that software.  We'd also like it to process a
simple dependency list so it can install all the software that a
project might need as well.

More to come...
