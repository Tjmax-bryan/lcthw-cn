\chapter{Exercise 7: More Variables, Some Math}

Let's get familiar with more things you can do with variables by 
declaring various \ident{ints}, \ident{floats}, \ident{chars}, and 
\ident{doubles}.  We'll then use these in various math expressions
so you get introduced to C's basic math.

\begin{code}{ex7.c}
<< d['code/ex7.c|pyg|l'] >>
\end{code}

Here's what's going on in this little bit of nonsense:

\begin{description}
\item[ex7.c:1-4] The usual start of a C program.
\item[ex7.c:5-6] Declare an \ident{int} and \ident{double} for some fake bug data.
\item[ex7.c:8-9] Print out those two, so nothing new here.
\item[ex7.c:11] Declare a huge number using a new type \ident{long} for storing
    big numbers.
\item[ex7.c:12-13] Print out that number using \verb|%ld| which adds a modifier
    to the usual \verb|%d|. Adding 'l' (the letter ell) means "print this
    as a long decimal".
\item[ex7.c:15-17] Just more math and printing.
\item[ex7.c:19-21] Craft up a depiction of your bug rate compared to the bugs in
    the universe, which is a completely inaccurate calculation.  It's so small
    though that we have to use \verb|%e| to print it in scientific notation.
\item[ex7.c:24] Make a character, with a special syntax \verb|'\0'| which creates
    a 'nul byte' character.  This is effectively the number 0.
\item[ex7.c:25] Multiply bugs by this character, which produces 0 for how much you
    should care.  This demonstrates an ugly hack you find sometimes.
\item[ex7.c:26-27] Print that out, and notice I've got a \verb|%%| (two percent
        chars) so I can print a '\%' (percent) character.
\item[ex7.c:28-30] The end of the \ident{main} function.
\end{description}

This bit of source is entirely just an exercise, and demonstrates how
some math works.  At the end, it also demonstrates something you
see in C, but not in many other languages.  To C, a "character" is just an
integer.  It's a really small integer, but that's all it is.  This means
you can do math on them, and a lot of software does just that, for good
or bad.

This last bit is your first glance at how C gives you direct access to 
the machine.  We'll be exploring that more in later exercises.


\section{What You Should See}

As usual, here's what you should see for the output:

\begin{code}{ex7 output}
\begin{lstlisting}
<< d['code/ex7.out|dexy'] >>
\end{lstlisting}
\end{code}


\section{How To Break It}

Again, go through this and try breaking the \ident{printf} by passing in
the wrong arguments.  See what happens when you try to print out that
\ident{nul\_byte} variable too with \verb|%s| vs. \verb|%c|.  When you
break it, run it under \ident{Valgrind} to see what it says about your
breaking attempts.

\section{Extra Credit}

\begin{enumerate}
\item Make the number you assign to \ident{universe\_of\_defects} various 
    sizes until you get a warning from the compiler.
\item What do these really huge numbers actually print out?
\item Change \ident{long} to \ident{unsigned long} and try to find 
    the number that makes that one too big.
\item Go search online to find out what \ident{unsigned} does.
\item Try to explain to yourself (before I do in the next exercise)
    why you can multiple a \ident{char} and an \ident{int}.
\end{enumerate}

