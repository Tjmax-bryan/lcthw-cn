\chapter{Introduction: The Cartesian Dream Of C}

\begin{quotation}

Whatever I have up till now accepted as most true and assured I have gotten
either from the senses or through the senses. But from time to time I have
found that the senses deceive, and it is prudent never to trust completely
those who have deceived us even once.

    \attrib{Rene Descartes, Meditations On First Philosophy}
\end{quotation}

If there ever were a quote that described programming with C, it would be this.
To many programmers, this makes C scary and evil.  It is the Devil, Satan, the
trickster Loki come to destroy your productivity with his seductive talk of
pointers and direct access to the machine.  Then, once this computational
Lucifer has you hooked, he destroys your world with the evil "segfault" and
laughs as he reveals the trickery in your bargain with him.

But, C is not to blame for this state of affairs.  No my friends, your computer
and the Operating System controlling it are the real tricksters.  They conspire
to hide their true inner workings from you so that you can never really know
what is going on.  The C programming language's only failing is giving you
access to what is really there, and telling you the cold hard raw truth.  C
gives you the red pill.  C pulls the curtain back to show you the wizard.
\emph{C is truth.}

Why use C then if it's so dangerous?   Because C gives you power over the false
reality of abstraction and liberates you from stupidity.


\section{What You Will Learn}

The purpose of this book is to get you strong enough in C
that you'll be able to write your own software in it, or modify
someone else's code.  At the end of the book we actually take
code from a more famous book called \krc and code review it
using what you've learned.  To get to this stage you'll have to
learn a few things:

\begin{enumerate}
\item The basics of C syntax and idioms.
\item Compilation, make files, linkers.
\item Finding bugs and preventing them.
\item Defensive coding practices.
\item Breaking C code.
\item Writing basic Unix systems software.
\end{enumerate}

By the final chapter you will have more than enough ammunition
to tackle basic systems software, libraries, and other smaller
projects.


\section{How To Read This Book}

This book is intended for programmers who have learned at least one other
programming language.  I refer you to
\href{http://learnpythonthehardway.org}{Learn Python The Hard Way} or to
\href{http://ruby.learncodethehardway.org}{Learn Ruby The Hard Way} if you
haven't learned a programming language yet.  Those two books are for total
beginners and work very well.  Once you've done those then you can come back
and start this book.

For those who've already learned to code, this book may seem strange
at first.  It's not like other books where you read paragraph after
paragraph of prose and then type in a bit of code here and there.  Instead
I have you coding right away and then I explain what you just did.
This works better because it's easier to explain something you've 
already experienced.

Because of this structure, there are a few rules you \emph{must} follow
in this book:

\begin{enumerate}
\item Type in all of the code. Do not copy-paste!
\item Type the code in exactly, even the comments.
\item Get it to run and make sure it prints the same output.
\item If there are bugs fix them.
\item Do the extra credit but it's alright to skip ones you can't figure out.
\item Always try to figure it out first before trying to get help.
\end{enumerate}

If you follow these rules, do everything in the book, and still can't
code C then you at least tried.  It's not for everyone, but the act
of trying will make you a better programmer.

\section{About "The Hard Way"}

I wrote about the philosophy of "The Hard Way" in my first book using
this method, but I'll paraphrase it again since I'm assuming you've
read my other beginner books or know what you're doing.

The key to "The Hard Way" isn't that this material is difficult, but
instead that it's "hard" because to be good at things you have to
practice.  Practice is sometimes painful, annoying, and tedious.  You
have to work and sweat to get what seems like absolutely no benefits
in the beginning.  But, if you work at it and focus on learning as
exactly as possible, then one day suddenly you can do it.

This book is organized so that, in the first section, you are doing
what seems like tedious boring irritating rote work.  This section
is all about making you strong and getting you the foundation skills.
The second half of the book is about applying those skills to more
fun problems and exploring all the things the language has to offer.
You might think you can skip the foundational skills at the beginning,
but if you do then you'll be missing out on gaining confidence and
building your "chops".

One tip for you if you find yourself really irritated at a particular
exercise is to \emph{take breaks}.  When you feel that annoyance build
up, just take a break and watch some TV or play video games.  I go play
guitar.  After a little bit, come back and keep going.  Even if you just
do it in 5 minute spurts, eventually you'll build up your stamina and
be able to write code for hours on end, lost in it.


\section{License}

This book is free for you to read, but until I'm done you can't distribute it
or modify it.  I need to make sure that unfinished copies of it do not get out
and mess up a student on accident.


